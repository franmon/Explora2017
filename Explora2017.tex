%%%%%%%%%%%%%%%%%%%%%%%%%%%%%%%%%%%%%%%%%%%%%%%%%%%%%%%%%%%%
\documentclass[a4paper,11pt,oneside]{article}
\usepackage[a4paper,vmargin={1.5cm,1.5cm},width=16cm]{geometry}
\usepackage[style=verbose-inote,doi=false,sortcites=true,block=space,backend=bibtex]{biblatex}
\usepackage[utf8]{inputenc}
\usepackage{textcomp}
\usepackage[spanish]{babel}
\usepackage{microtype}
\usepackage{lmodern}
\usepackage{graphicx}
\usepackage{fancyhdr}
\usepackage{booktabs}
\usepackage{eurosym}
\usepackage{mathptmx}
\usepackage[T1]{fontenc}
\usepackage{hyperref}
\usepackage{parskip}
%% Added to help mimic structure.
\usepackage[many]{tcolorbox}
\usepackage{soul}
\usepackage{color}
\usepackage{lastpage}

%Arial font
\renewcommand{\rmdefault}{phv} % Arial
\renewcommand{\sfdefault}{phv} % Arial

%\usepackage[skins]{tcolorbox}
%%%%%%%%%%%%%%%%%%%%%%%%%%%%%%%%%%%%%%%%%%%%%%%%%%%%%%%%%%%%
%% HEADERS
%\setlength{\headheight}{1cm}
%\setlength{\headsep}{0.5cm}
%\pagestyle{fancyplain}
%\fancyhf{}
%\lhead{\fancyplain{}{\sc Memoria científico técnica de proyectos coordinados}}
%\rhead{\fancyplain{}{\sc Parte A}}
%\cfoot{\thepage}
%\renewcommand{\headrulewidth}{0pt} % remove lines
%\renewcommand{\footrulewidth}{0pt}


%%% HEADER
\setlength{\headheight}{1cm}
%\setlength{\headwidth}{20cm}
\setlength{\headsep}{0.5cm}
\pagestyle{fancyplain}
\fancyheadoffset[HR,HL]{2cm}
\fancyhf{}
\lhead{\raisebox{-0.4\height}{\includegraphics[height=0.9cm,keepaspectratio=true]{img/miniLogo}}}
\rhead{\fancyplain{}{\fontsize{10}{12} \selectfont \textbf{\underline{MEMORIA CIENT\'IFICO-T\'ECNICA DE PROYECTOS EXPLORA}}}}
\cfoot{\thepage\, / parte A}
\renewcommand{\headrulewidth}{0pt} % remove lines
\renewcommand{\footrulewidth}{0pt}
%%%%%%%%%%%%%%%%%%%%%%%%%%%%%%%%%%%%%%%%%%%%%%%%%%%%%%%%%%%%
%% Hack to make math formulas bold in section titles
\makeatletter
\DeclareRobustCommand*{\bfseries}{%
  \not@math@alphabet\bfseries\mathbf
  \fontseries\bfdefault\selectfont
  \boldmath
}
\makeatother

%%%%%%%%%%%%%%%%%%%%%%%%%%%%%%%%%%%%%%%%%%%%%%%%%%%%%%%%%%%%
\def\thesection{\bf \textsf{\Alph{section}}}

%\nobibliography{biblio}
%\bibliographystyle{JHEP}

\bibliography{biblio}


\def\changemargin#1#2{\list{}{\rightmargin#2\leftmargin#1}\item[]}
\let\endchangemargin=\endlist 

%%%%%%%%%%%%%%%%%%%%%%%%%%%%%%%%%%%%%%%%%%%%%%%%%%%%%%%%%%%%
\begin{document}

%% Some useful definitions
\input{src/Commands.tex}

%% Heading
\begin{center}
\begin{tcolorbox}[width=12cm, colback=white,arc=0pt,outer arc=0pt,colframe=black,boxrule=0.6pt]
\centering
Convocatoria 2017\\
Proyectos «Explora Ciencia» y «Explora Tecnolog\'ia» \\
Direcci\'on General de Investigaci\'on Cient\'ifica y T\'ecnica\\
Subdirecci\'on General de Proyectos de Investigaci\'on

%
%Convocatorias 2015\\ 
%Proyectos Excelencia y Proyectos RETOS \\ 
%Dirección General de Investigación Científica y Técnica \\
%Subdirección General de Proyectos de Investigación
\end{tcolorbox}
\end{center} 

\begin{tcolorbox}[width=16.7cm,colback=yellow,arc=0pt,outer arc=0pt,colframe=black,boxrule=0.6pt,left=0mm,right=0mm, boxsep=2mm, bottom=1mm]
  \begin{center}
    \textbf{AVISO IMPORTANTE}
  \end{center}

\vspace{0.5cm}
En virtud del art\'iculo 11 de la convocatoria \ul{\textbf{NO SE ACEPTAR\'AN NI SER\'AN SUBSANABLES MEMORIAS CIENT\'IFICO-T\'ECNICAS}} que no se presenten en este formato.
     \\
     \\
    \textbf{La parte C no podrá exceder 10 p\'aginas.}
    \\
    \\
    \textbf{Lea detenidamente las instrucciones que figuran al final de este documento para rellenar correctamente la memoria cient\'ifico-t\'ecnica.}
  %\end{center}
\end{tcolorbox}
\vspace{3pt}
\begin{tcolorbox}[width=14.5cm,colback=yellow,arc=0pt,outer arc=0pt,colframe=black,boxrule=0.6pt,left=0mm, boxsep=2mm]
  \noindent\textbf{Parte A: RESUMEN DE LA PROPUESTA/SUMMARY OF THE PROPOSAL}
  %\section{RESUMEN DE LA PROPUESTA/SUMMARY OF THE PROPOSAL}
\end{tcolorbox}

%%%%%%%%%%%%%%%%%%%%%%%%%%%%%%%%%%%%%%%%%%%%%%%%%%%%%%%%%%%%%%%%%%%%%%%%%%%%%

%\section{RESUMEN DE LA PROPUESTA/SUMMARY OF THE PROPOSAL}

%%%%%%%%%%%%%%%%%%%%%%%%%%%%%%%%%%%%%%%%%%%%%%%%%%%%%%%%%%%%%%%%%%%%%%%%%%%%%

%\subsection{DATOS DEL PROYECTO COORDINADO}
%\noindent\textbf{A.1. DATOS DEL PROYECTO COORDINADO}
\vspace{6pt}

\noindent\textbf{INVESTIGADOR PRINCIPAL:} (Nombre y apellidos)

\noindent Juan Jos\'e Gomez Cadenas.
\vspace{6pt}


\noindent 
\vspace{6pt}

\noindent\textbf{TÍTULO DEL PROYECTO:} Cold Xenon gas for neutrinoless double beta searches.
%Espectroscop\'ia de gas xenon cerca del punto de liquefacci\'on.
\vspace{6pt}

\noindent\textbf{ACR\'ONIMO} CoXe0nu


\noindent\textbf{RESUMEN} 
{\color{blue}{M\'aximo 2000 caracteres (incluyendo espacios en blanco):}}
\vspace{12pt}


\noindent\textbf{PALABRAS CLAVE} 


\noindent\textbf{TITLE OF THE PROJECT:} 
\vspace{6pt}

\noindent\textbf{ACRONYM:} .
\vspace{6pt}

\noindent\textbf{SUMMARY} 
{\color{blue}{ Maximum 2000 characters (including spaces):}}
\vspace{6pt}

\newpage

\setcounter{page}{1}
\cfoot{\fancyplain{}{\thepage\, de \pageref{LastPage} / parte B}}

%Autoevaluacion
\begin{tcolorbox}[width=16.7cm,colback=yellow,arc=0pt,outer arc=0pt,colframe=black,boxrule=0.6pt,left=0mm, boxsep=0mm, left=2mm]
  \noindent\textbf{Parte B: AUTOEVALUACI\'ON E INFORMACI\'ON ESPEC\'IFICA DEL EQUIPO DE TRABAJO}
%  \noindent Parte B: AUTOEVALUACI\'ON E INFORMACI\'ON ESPEC\'IFICA DEL EQUIPO DE TRABAJO
  %\section{RESUMEN DE LA PROPUESTA/SUMMARY OF THE PROPOSAL}
\end{tcolorbox}

\begin{changemargin}{0.5cm}{0.5cm} 

\noindent\textbf{B.1. AUTOEVALUACI\'ON} Responda de forma escueta a las siguientes preguntas:
\vspace{0.75cm}

\noindent\textbf{1. ¿Cu\'al o cu\'ales son, en su opini\'on, los puntos m\'as importantes por los que su
propuesta deber\'ia ser financiada?}
\newline{\color{blue}{M\'aximo 500 caracteres}}
\vspace{2cm}

\noindent\textbf{2. ¿Cu\'al o cu\'ales son, en su opini\'on, los puntos m\'as d\'ebiles de su propuesta? En otras palabras ¿por qu\'e su solicitud ser\'ia rechazada en una convocatoria cl\'asica de otros programas?}
\newline{\color{blue}{M\'aximo 500 caracteres}}
\vspace{2cm}

\noindent\noindent\textbf{3. Resalte los aspectos de su experiencia investigadora que demuestren que est\'a usted (y, si es el caso, su equipo) preparado para afrontar este estudio.}\newline{\color{blue}{M\'aximo 500 caracteres}}
\vspace{2cm}

\noindent \textbf{4. Indique si ha presentado esta propuesta, u otra de contenido similar, a alguna convocatoria de proyectos de investigaci\'on.
}\newline{\color{blue}{M\'aximo 500 caracteres}}
\vspace{2cm}

\noindent\textbf{5. Cu\'al es el grado de confidencialidad que solicita para la evaluaci\'on de su propuesta?}
\newline
\vspace{1cm}

\noindent\textbf{Muy alto\hspace{1cm}; Alto X; \hspace{1cm} Normal \hspace{1cm};No es relevante}
\vspace{2cm}

\textbf{6. Justificaci\'on del presupuesto} \newline{\color{blue}{M\'aximo 2000 caracteres}}
\vspace{2cm}


%\restoregeometry
\noindent\textbf{B.2. RELACI\'ON DE LAS PERSONAS NO DOCTORES QUE COMPONEN EL EQUIPO DE TRABAJO }(se recuerda que los doctores del equipo de trabajo y los componentes del equipo de investigaci\'on no se solicitan aqu\'i porque deber\'an incluirse en la aplicaci\'on inform\'atica de solicitud). Repita la siguiente secuencia tantas veces como precise.
\\\vspace{0.5cm}
\noindent 1. Nombre y apellidos:\\[-0.5cm]
\setlength{\parindent}{15pt} 
\indent Titulaci\'on: licenciado/ingeniero/graduado/m\'aster\/ formaci\'on profesional/otros (especificar)

Tipo de contrato: en formaci\'on/contratado/t\'ecnico/entidad extranjera/otros (especificar) 

Duraci\'on del contrato: indefinido/temporal


\end{changemargin}

 
\vspace{12pt}

\newpage
\setcounter{page}{1}
\cfoot{\fancyplain{}{\thepage\, de \pageref{LastPage} / parte C}}

%%%%%%%%%%%%%%%%%%%%%%%%%%%%%%%%%%%%%%%%%%%%%%%%%%%%%%%%%%%%%%%%%%%%%%%%%%%%%

%\section{DOCUMENTO CIENTÍFICO / SCIENTIFIC DOCUMENT}
\newpage
\begin{tcolorbox}[colback=yellow,arc=0pt,outer arc=0pt,colframe=black,boxrule=0.6pt,left=0mm]
  \noindent\textbf{Parte C: DOCUMENTO CIENTÍFICO / SCIENTIFIC DOCUMENT: Máximo 10 páginas}
  %\section{RESUMEN DE LA PROPUESTA/SUMMARY OF THE PROPOSAL}
\end{tcolorbox}

\subsubsection*{C.1. PROPUESTA CIENT\'IFICA}

\subsubsection*{C.2. IMPACTO CIENT\'IFICO Y/O SOCIO-ECON\'OMICO DEL PROYECTO EN CASO DE TENER \'EXITO}

\subsubsection*{C.3. PERSONAL INVESTIGADOR QUE PARTICIPAR\'A EN EL PROYECTO Y DESCRIPCI\'ON DEL PAPEL DE CADA UNO DE ELLOS}

\subsubsection*{C.4. IMPLICACIONES \'ETICAS Y/O DE BIOSEGURIDAD}




\end{document}

